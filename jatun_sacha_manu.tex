\documentclass[12pt]{report} % change if you want, e.g., smaller fonts
\usepackage{%
graphicx,
hyperref,
geometry,
url,
parskip,
PTSerif} % change the font package if needed

% Page size (A5), change it to the size you need
\geometry{paperheight=210mm,paperwidth=148mm,margin=14mm,top=26mm}

% ===

% Define sections
\newcommand{\FF}[1]{\section{#1}}
\renewcommand{\thesection}{\arabic{section}} % there are no chapters

% Define species names
\newcommand{\KK}[1]{\textit{#1}}
\newcommand{\SP}[1]{{\bigskip\noindent\textbf{#1}\par\nopagebreak[4]\smallskip}}

% Define images
\newcommand{\I}[1]{\includegraphics[width=.95\textwidth]{images/thumbs/#1}\quad}
\newcommand{\II}[1]{{\nopagebreak\medskip\raggedright\noindent\leftskip1.2em #1\par\bigskip}}

% Define text descriptions
\newcommand{\DD}[1]{{\noindent\leftskip1.2em\small\textbf{Description:} #1.\par}}

% ===

% Now make title page
\title{Plants of Jatun Sacha, Ecuador}
\author{Illustrated checklist of the most collected plants}
\date{\today}

\begin{document}

\maketitle

\tableofcontents

\sloppy % useful because it is hard to control "0body" manually

\newpage

\section*{Introduction}

Plants were selected to this manual from the checklist available here at
\url{http://www.mobot.org/MOBOT/research/ecuador/jatun/checklist.shtml}

This checklist was processed with TNRS
(\url{http://tnrs.iplantcollaborative.org/TNRSapp.html}) to unify names
and authorships and also obtain names of families. Next, R script
determined which plants were collected most. I choosed 100 plants
collected more than 11 times plus some random, less collected plants.
Next, photographs of each species were obtained\footnote{This checklist
is intended exclusively for the educational, non-commercial use.}, and
converted into smaller \emph{rectangular} thumbnails.

Finally, the \textsf{Rmanual} artificial example was taken as source, and
I started to make this checklist. This making process included (1)
creation of tables, (2) tuning the \textsf{R} script and (3) running
\TeX\ and tuning the resulted PDF. The total time spent on this stage was
1 hour and 45 minutes.

Then a took a libery to enchance this variant of \textsf{Rmanual} and
added the possibility to insert species descriptions, each as separate
text file from \texttt{txts} directory. It was done only for one species,
but it is possible now to add descriptions for all listed species.

This manual is intended exclusively for educational, non-commercial use.
All rights belong to the copyright holder(s).

\bigskip

\hfill\emph{Alexey Shipunov}

\newpage

\input 0body

\end{document}
